%% LyX 1.3 created this file.  For more info, see http://www.lyx.org/.
%% Do not edit unless you really know what you are doing.
\documentclass[english,12pt,a4paper,twoside]{book}
\usepackage{times}
%\usepackage{algorithm2e}
\usepackage{url}
\usepackage{bbm}
\usepackage[T1]{fontenc}
\usepackage[latin1]{inputenc}
\usepackage{geometry}
%\geometry{verbose,letterpaper,tmargin=2.5cm,bmargin=3cm,lmargin=3cm,rmargin=3cm}
\usepackage{rotating}
\usepackage{graphicx}
\usepackage{amsmath, amsthm, amssymb}
\usepackage{setspace}
\usepackage{lineno}
\usepackage{hyperref}
\usepackage{bbm}
\usepackage{pdfpages}

%\usepackage{xcolor,framed}
%\colorlet{shadecolor}{blue!10}
%\begin{shaded}blabla\end{shaded}

%\usepackage{xr}
%\externaldocument{PRS-supp}

%\linenumbers
%\doublespacing
\onehalfspacing
\usepackage[authoryear]{natbib}
%\usepackage{natbib}
%\usepackage{chapterbib}

\usepackage{numprint}
\npthousandsep{,}

%Pour les rajouts
\usepackage{xcolor}

\usepackage{dsfont}
\usepackage[warn]{textcomp}
\usepackage{adjustbox}
\usepackage{multirow}
\usepackage{graphicx}
\graphicspath{{figures/}}
\DeclareMathOperator*{\argmin}{\arg\!\min}

\let\tabbeg\tabular
\let\tabend\endtabular
\renewenvironment{tabular}{\begin{adjustbox}{max width=0.75\textwidth}\tabbeg}{\tabend\end{adjustbox}}

\makeatletter

\usepackage{afterpage}

\newcommand\blankpage{%
    \null
    \thispagestyle{empty}%
    \addtocounter{page}{-1}%
    \newpage}

%%%%%%%%%%%%%%%%%%%%%%%%%%%%%% LyX specific LaTeX commands.
%% Bold symbol macro for standard LaTeX users
%\newcommand{\boldsymbol}[1]{\mbox{\boldmath $#1$}}

%% Because html converters don't know tabularnewline
\providecommand{\tabularnewline}{\\}
\definecolor{clumping}{HTML}{38761D}
\definecolor{thresholding}{HTML}{1515FF}
%<span style="color:#38761D">Clumping</span> + <span style="color:#1515FF">Thresholding</span>

\usepackage{babel}
\makeatother


\begin{document}

\includepdfset{offset=.15in 0in,noautoscale,scale=1,pages={-},pagecommand={}}
\includepdf{cover}

\afterpage{\blankpage}


\clearpage

\section*{Abstract}

Genotyping is becoming cheaper, making genotype data available for millions of individuals.
Moreover, imputation enables to get genotype information at millions of loci capturing most of the genetic variation in the human genome.
Given such large data and the fact that many traits and diseases are heritable (e.g.\ 80\% of the variation of height in the population can be explained by genetics), it is envisioned that predictive models based on genetic information will be part of a personalized medicine.

In my thesis work, I focused on improving predictive ability of polygenic models.
Because prediction modeling is part of a larger statistical analysis of datasets, I developed tools to allow flexible exploratory analyses of large datasets, which consist in two R/C++ packages described in the first part of my thesis.
Then, I developed some efficient implementation of penalized regression to build polygenic models based on hundreds of thousands of genotyped individuals.
Finally, I improved the ``clumping and thresholding'' method, which is the most widely used polygenic method and is based on summary statistics that are widely available as compared to individual-level data.

Overall, I applied many concepts of statistical learning to genetic data. I used extreme gradient boosting for imputing genotyped variants, feature engineering to capture recessive and dominant effects in penalized regression, and parameter tuning and stacked regressions to improve polygenic prediction.
Statistical learning is not widely used in human genetics and my thesis is an attempt to change that.


\clearpage

\section*{R\'esum\'e}

Le g\'enotypage devient de moins en moins cher, rendant les donn\'ees de g\'enotypes disponibles pour des millions d'individus.
Par ailleurs, l'imputation permet d'obtenir l'information  g\'enotypique pour des millions de positions de l'ADN, capturant l'essentiel de la variation g\'en\'etique du g\'enome humain.
Compte tenu de la richesse des donn\'ees et du fait que de nombreux traits et maladies sont h\'er\'editaires (par exemple, la g\'en\'etique peut expliquer 80\% de la variation de la taille dans la population), il est envisag\'e d'utiliser des mod\`eles pr\'edictifs bas\'es sur l'information g\'en\'etique dans le cadre d'une m\'edecine personnalis\'ee.

Au cours de ma th\`ese, je me suis concentr\'e sur l'am\'elioration de la capacit\'e pr\'edictive des mod\`eles polyg\'eniques.
Les mod\`eles pr\'edictifs faisant partie d'une analyse statistique plus large des jeux de donn\'ees, j'ai d\'evelopp\'e des outils permettant l'analyse exploratoire de grands jeux de donn\'ees, constitu\'es de deux packages R/C++ d\'ecrits dans la premi\`ere partie de ma th\`ese.
Ensuite, j'ai d\'evelopp\'e une impl\'ementation efficace de la r\'egression p\'enalis\'ee pour construire des mod\`eles polyg\'eniques bas\'es sur des centaines de milliers d'individus g\'enotyp\'es.
Enfin, j'ai am\'elior\'e la m\'ethode appel\'ee ``clumping and thresholding'', qui est la m\'ethode polyg\'enique la plus largement utilis\'ee et qui est bas\'ee sur des statistiques r\'esum\'ees plus largement accessibles par rapport aux donn\'ees individuelles.

Dans l'ensemble, j'ai appliqu\'e de nombreux concepts d'apprentissage statistique aux donn\'ees g\'en\'etiques. J'ai utilis\'e du ``extreme gradient boosting'' pour imputer des variants g\'enotyp\'es, du ``feature engineering'' pour capturer des effets r\'ecessifs et dominants dans une r\'egression p\'enalis\'ee, et du ``parameter tuning'' et des ``stacked regressions'' pour am\'eliorer les mod\`eles polyg\'eniques pr\'edictifs.
L'apprentissage statistique n'est pour l'instant pas tr\`es utilis\'e en g\'en\'etique humaine et ma th\`ese est une tentative pour changer cela.


\clearpage

\section*{Remerciements}

Je voudrais d'abord remercier le LabEx PERSYVAL-Lab, l'universit\'{e} Grenoble Alpes, l'\'{e}cole doctorale EDISCE et le laboratoire TIMC-IMAG pour m'avoir permis de faire cette th\`{e}se. J'aimerais ensuite remercier les membres du jury, Florence, Julien, Benoit et Laurent pour avoir accept\'{e} de faire partie du jury de th\`{e}se, s'\^{e}tre d\'{e}plac\'{e} pour ma soutenance et s'\^{e}tre int\'{e}ress\'{e} \`{a} mon travail de th\`{e}se. J'aimerais aussi remercier les membres de mon comit\'{e} de suivi de th\`{e}se, Thomas et Julien, pour avoir suivi mes travaux de th\`{e}se et assur\'{e} que je gardais un bon cap. 

J'aimerais remercier \'{e}galement mes encadrants de th\`{e}se, Michael et Hugues, pour m'avoir accompagn\'{e} lors de cette th\`{e}se. J'ai parfois \'{e}t\'{e} dur en n\'{e}gociations. Michael m'a beaucoup apport\'{e} au point de vue communication, comment \'{e}crire un papier, comment articuler une pr\'{e}sentation, comment faire un beau tweet. Quant \`{a} Hugues, on ne s'est pas beaucoup vus mais il a pu apporter un regard neuf souvent utile. Je le remercie aussi de m'avoir mis en contact avec Bjarni avec qui j'ai pass\'{e} quelques mois \`{a} Aarhus au Danemark en tant que doctorant visiteur, et o\`{u} je vais continuer un postdoc pour les deux prochaines ann\'{e}es.

J'aimerais aussi remercier tous les membres de l'\'{e}quipe et du laboratoire pour leur bon humeur. Cela va de m\^{e}me pour l'\'{e}quipe de Hugues \`{a} l'Institut Pasteur, ainsi que les chercheurs \`{a} Aarhus au Danemark. Particuli\`{e}rement, Keurcien pour, entre autres, les soir\'{e}es "bi\`{e}re, saucisson, macdo, chartreuse". J'aimerais remercier Nicolas pour sa mise \`{a} disposition et sa gestion des serveurs de l'\'{e}quipe, que j'ai beaucoup utilis\'{e}s pendant la derni\`{e}re ann\'{e}e de ma th\`{e}se. J'aimerais aussi remercier Magali et Michael pour m'avoir aid\'{e} \`{a} d\'{e}marrer le groupe R de Grenoble. Cela fait d\'{e}j\`{a} maintenant deux ans qu'il tourne, merci \`{a} tous ceux qui ont pr\'{e}sent\'{e} et particip\'{e}, et merci \`{a} Matthieu d'avoir repris le flambeau quand je suis parti au Danemark.

Enfin, merci \`{a} ma ch\'{e}rie, Sylvie, de m'avoir support\'{e} tout au long de cette th\`{e}se. J'ai eu un gros coup de mou \`{a} la moiti\'{e}. Aussi, les \textquotedblleft{}bon j'en ai marre de travailler, je vais faire la sieste\textquotedblright{} \`{a} 15h alors qu'elle travaillait jusqu'\`{a} 18h30, ce n'\'{e}tait pas forc\'{e}ment tr\`{e}s sympa. Et merci d'avoir accept\'{e} de partir au Danemark avec moi. Un dernier mot pour mes parents, pas forc\'{e}ment pour la th\`{e}se, mais pour tout mon parcours scolaire. Ils ont toujours fait en sorte que je ne manque rien, ce qui m'a permis de faire mes \'{e}tudes dans les meilleures conditions possibles. Merci.




\clearpage

\tableofcontents


\clearpage

\chapter{Introduction}


\includepdfset{offset=-.25in 0in,noautoscale,scale=0.83,pages={-},pagecommand={}}
%% LyX 1.3 created this file.  For more info, see http://www.lyx.org/.
%% Do not edit unless you really know what you are doing.
\documentclass[english, 12pt]{article}
\usepackage{times}
%\usepackage{algorithm2e}
\usepackage{url}
\usepackage{bbm}
\usepackage[T1]{fontenc}
\usepackage[latin1]{inputenc}
\usepackage{geometry}
\geometry{verbose,letterpaper,tmargin=2.5cm,bmargin=2.5cm,lmargin=2.5cm,rmargin=2.5cm}
\usepackage{rotating}
\usepackage{graphicx}
\usepackage{amsmath, amsthm, amssymb}
\usepackage{setspace}
\usepackage{lineno}
\usepackage{hyperref}
\usepackage{bbm}

%\usepackage{xcolor,framed}
%\colorlet{shadecolor}{blue!10}
%\begin{shaded}blabla\end{shaded}


%\usepackage{xr}
%\externaldocument{PRS-supp}

\linenumbers
\doublespacing
%\usepackage[authoryear]{natbib}
\usepackage{natbib} \bibpunct{(}{)}{;}{author-year}{}{,}

%Pour les rajouts
\usepackage{xcolor}

\usepackage{dsfont}
\usepackage[warn]{textcomp}
\usepackage{adjustbox}
\usepackage{multirow}
\usepackage{graphicx}
\graphicspath{{figures/}}
\DeclareMathOperator*{\argmin}{\arg\!\min}

\let\tabbeg\tabular
\let\tabend\endtabular
\renewenvironment{tabular}{\begin{adjustbox}{max width=\textwidth}\tabbeg}{\tabend\end{adjustbox}}

\makeatletter

%%%%%%%%%%%%%%%%%%%%%%%%%%%%%% LyX specific LaTeX commands.
%% Bold symbol macro for standard LaTeX users
%\newcommand{\boldsymbol}[1]{\mbox{\boldmath $#1$}}

%% Because html converters don't know tabularnewline
\providecommand{\tabularnewline}{\\}
\definecolor{clumping}{HTML}{38761D}
\definecolor{thresholding}{HTML}{1515FF}
%<span style="color:#38761D">Clumping</span> + <span style="color:#1515FF">Thresholding</span>

\usepackage{babel}
\makeatother


\begin{document}

\section{Introduction}

In my thesis work, we have been focusing on assessing someone's risk of disease based on DNA mutations data. DNA mutations do not really change over lifetime so that we could, in theory, assess someone's risk of disease at birth. Thus, this could have potentially large implications in disease prevention. As an example, about 12\% of women in the general population will develop breast cancer sometime during their lives \cite[]{desantis2016breast}. By contrast, a recent large study estimated that about 72\% (95\% CI: 65\%-79\%) of women who inherit a harmful BRCA1 mutation and about 69\% (95\% CI: 61\%-77\%) of women who inherit a harmful BRCA2 mutation will develop breast cancer by the age of 80 \cite[]{kuchenbaecker2017risks}. In 2013, Angelina Jolie announced that she had undergone a preventative double mastectomy, because she had a family history of breast cancer and was carrying a harmful BRCA1 mutation.

In this introduction, [TODO]

\subsection{[SOMEWHERE?]}

Selection might also be responsible for keeping genetic effect sizes low, since variants of larger effect may be selected against and eventually disappear \cite[]{pritchard2002allelic}.

\subsection{Context}

Today, clinical risk prediction for common adult-onset diseases often relies on basic demographic characteristics, such as age, gender and ethnicity; basic health parameters and lifestyle factors, such as body mass index, smoking status, alcohol consumption and physical exercise habits; measurement of clinical risk factors proximal to overt disease onset, such as blood pressure levels, blood chemistries or biomarkers indicative of ongoing disease processes; ascertainment of environmental exposures, such as air pollution, heavy metals and other environmental toxins; and family history \cite[]{torkamani2018personal}.
Routine genetic profiling is conspicuously absent from this list, often relegated to use only when testing clarifies individual-level risks in the context of a known family history for some common adult-onset
diseases \cite[]{torkamani2018personal}.

\subsubsection{Different types of disease and mutations}

Different types of mutations exist (Figure \ref{fig:rare-common}) [PARLER DE DISEASE ARCHITECTURE??].
Along with harmful BRCA mutations, many other highly penetrant mutations\footnote{most people carrying the mutation will develop the disease} are associated with diseases, and are searchable in an online database \cite[]{hamosh2005online}. Those mutation are often very rare and are either associated with some very rare disease or are explaining only a small proportion of common disease incidence.
In this work, we focus on common diseases (e.g.\ breast cancer) and try to predict individuals' disease susceptibility based on common variants; the common disease--common variant hypothesis \cite[]{pritchard2002allelic}. This hypothesis further suggests that such diseases are likely caused by a large number of common variants, each contributing only a small risk and thereby evading negative evolutionary selection \cite[]{salari2012personalized}.
One common form of variation across human genomes is called a single nucleotide polymorphism (SNP). As indicated by the name, SNPs are single base changes in the DNA.
Sequencing technologies now exists to genotype hundreds of thousands of variants at once for around \$50 only. Starting with the \cite{wellcome2007genome}, these new sequencing technologies had led to many genome-wide association studies.%, which we talk about in the next section. From these studies, it was found that effects of common variants that are associated with some disease are typically very small.

\begin{figure}[htb]
\centerline{\includegraphics[width=0.7\textwidth]{rare-common.jpg}}
\caption{Feasibility of identifying genetic variants by risk allele frequency and strength of genetic effect (odds ratio). Most emphasis and interest lies
in identifying associations with characteristics shown within diagonal dotted
lines. Source: \cite{manolio2009finding}.}
\label{fig:rare-common}
\end{figure}

\subsubsection{Genome-Wide Association Studies (GWAS)}

\cite{visscher201710} provide a thorough review of the aims and outcomes of GWAS.
The method behind GWAS is simple: test each variant one by one for association with a phenotype of interest.
For a continuous phenotype (e.g.\ height), linear regression is used and a t-test is performed on $\beta^{(j)}$ for each SNP $j$ where
\begin{equation}
\hat{y} = \alpha^{(j)} + \beta^{(j)} SNP^{(j)} + \gamma_1^{(j)} COV_1 + \cdots + \gamma_K^{(j)} COV_K + \epsilon~,\label{eq:gwas1}
\end{equation}
$K$ is the number of covariates, including principal components and other covariates such as age and gender. Similarly, for a binary phenotype (e.g.\ disease status), logistic regression is used and a Z-test is performed on $\beta^{(j)}$ for each SNP $j$ where
\begin{equation}
\log{\left(\frac{\hat{p}}{1-\hat{p}}\right)} = \alpha^{(j)} + \beta^{(j)} SNP^{(j)} + \gamma_1^{(j)} COV_1 + \cdots + \gamma_K^{(j)} COV_K + \epsilon~,\label{eq:gwas2}
\end{equation}
$\hat{p} = \mathbb{P}(Y = 1)$ and $Y$ denotes the binary phenotype.
It is well established that principal components of genotype data should be included as covariates in GWAS \cite[]{price2006principal}. Indeed, principal components of genotype data capture well population structure (as shown in figure \ref{fig:pca}). 
For example, consider a dataset where you have 900 Finnish people and only 100 Italian people. Because Finnish people are taller than Italian people, any SNP with a large difference in allele frequency between these two populations would be flagged as being associated with height, leading to many false positive associations. Adding principal components as covariates aims at preventing those SNPs from being false positive reports [PARLER DE QQ-PLOT ET DE LAMBDAGC??].

\begin{figure}[htb]
\centerline{\includegraphics[width=0.6\textwidth]{celiac-pca.png}}
\caption{First two Principal Components of individuals from four European populations. PC1 correlates with latitude while PC2 correlates longitude.}\label{fig:pca}
\end{figure}

These simple tests can be used only if individuals are not related to one another. If they do, a common practice is to remove one individual from each pairs of related individuals. Another strategy is to use Linear Mixel Models (LMM) to take into account both relatedness and population structure; these mixed models have also the potential to increase discovery power \cite[]{yang2014advantages}.

To date, more than 10,000 strong associations have been reported between genetic variants and one or more complex traits \cite[]{welter2013nhgri}, where ``strong'' is defined as statistically significant at the genome-wide p-value threshold of $5 \cdot 10^{-8}$, which corresponds to a type-I error of 5\%, Bonferroni-corrected for one million independent tests \cite[]{pe2008estimation}. Results of a GWAS are usually reported in a Manhattan plot (Figure \ref{fig:gwas}). This type of plot shows some association peaks (similar to skyscrapers in Manhattan) due to some local correlation between SNPs (Linkage Disequilibrium), with squared correlation roughly inversely proportional to distance between SNPs \cite[]{hudson2001two}.

\begin{figure}[htb]
\centerline{\includegraphics[width=0.95\textwidth]{gwas-height-20K.png}}
\caption{Manhattan plot of GWAS of height based on 20,000 unrelated individuals.}\label{fig:gwas}
\end{figure}


\subsubsection{GWAS data}

There are mainly three types of data: genotyped SNPs from genotyping chips, imputed SNPs from reference panels and Next Generation Sequencing (NGS) data.
Genotyping chips enables a quick and cheap genotyping of 200K to 2M SNPs, mostly focusing on common variants (Minor Allele Frequency (MAF) larger than 5\%). From this genotyping, you can get a matrix of $0$s, $1$s and $2$s, counting the number of alternative alleles for each individual (row) and each genome position (column). There are usually few missing values (less than 5\% in total).

Imputation has a different meaning in genetics than in other Data Science fields; it does not refer to filling those 5\% missing values, but instead refers to adding completely new variants that were not genotyped with the chip used. 
Imputation is enabled by the fact that the genotypes of unobserved genetic variants can be predicted by the haplotypes inferred from multiple observed SNPs (the ones that were genotyped) and the haplotypes observed from a fully sequenced reference
panel \cite[]{marchini2010genotype,mccarthy2016reference}.
Imputation now allows to have datasets such as the UK Biobank: 90M SNPs for each of 500K individuals \cite[]{bycroft2017genome}.

Finally, NGS (also called Whole Genome Sequencing (WGS)) refers to fully sequenced data over more than 3M variants, including some rare variants. Yet, this technology is still very expensive, with a cost of around \$1000 per genome.
GWAS to date have been based on SNP arrays designed to tag common variants in the genome. These arrays do not cover all genetic variants in the population, and it seems natural that future GWAS will be based on WGS. However, the price differential between SNP arrays and WGS is still substantial, and array technology remains more robust than sequencing \cite[]{visscher201710}. An in-between solution could be to use extremely low-coverage sequencing \cite[]{pasaniuc2012extremely}.

Recently, some national biobanks projects have emerged. For example, the UK Biobank has released both genome-wide genotypes and rich phenotypic data on 500K individuals to the international research community \cite[]{bycroft2017genome}.
Yet, most of the time, only summary statistics for a GWAS dataset (estimated effect sizes and p-values of SNPs) are available (not the individual-level genotype data). Because of the availability of such data en masse, specific methods using those summary data have been developed \cite[]{pasaniuc2014fast,vilhjalmsson2015modeling,bulik2015ld,pasaniuc2017dissecting,speed2018sumher}. The craze for such data can be explained by the fact that GWAS individual-level data, sometimes consisting of a meta-analysis of many small datasets, cannot be easily shared publicly, as opposed to summary data. Moreover, methods using summary statistics data are usually fast and easy to use, making them even more appealing to researchers.

In this thesis, we used genotyped SNPs, imputed SNPs and summary statistics.


%%%%%%%%%%%%%%%%%%%%%%%%%%%%%%%%%%%%%%%%%%%%%%%%%%%%%%%%%%%%%%%%%%%%%%%%%%%%%%%%


\subsection{From GWAS to Polygenic Risk Scores (PRS)}

\subsubsection{A standard way to compute PRS}

The main method for computing Polygenic Risk Scores (PRS) is the widely used ``Clumping + Thresholding'' (C+T, also called ``Pruning + Thresholding'' in the literature) model based on univariate GWAS summary statistics as described in equations \eqref{eq:gwas1} and \eqref{eq:gwas2}.
Under the C+T model, a coefficient of regression is learned independently for each SNP along with a corresponding p-value (the GWAS part). The SNPs are first clumped (C) so that there remains only SNPs that are weakly correlated with each other ($S_\text{clumping}$). Thresholding (T) consists in removing SNPs that are under a certain level of significance (p-value threshold $p_T$ to be determined). A polygenic risk score is defined as the sum of allele counts of the remaining SNPs weighted by the corresponding regression coefficients \cite[]{purcell2009common,Dudbridge2013,Euesden2015},
\[\rm{PRS}_i = \sum_{\substack{j \in S_\text{clumping} \\ p_j~<~p_T}} \hat\beta_j \cdot G_{i,j}~,\] where $\hat\beta_j$ ($p_j$) are the effect sizes (p-values) learned from the GWAS and $G_{i,j}$ is the allele count (genotype) for individual $i$ and SNP $j$.

\begin{figure}[htb]
\centerline{\includegraphics[width=0.95\textwidth]{GWAS2PRS3.png}}
\caption{Illustration of C+T looking at a Manhattan plot of GWAS of height based on 20,000 unrelated individuals. \textbf{\color{clumping}Clumping:} due to Linkage Disequilibrium, indirect associations provides only redundant information (see Figure \ref{fig:gwasLD}). \textbf{\color{thresholding}Thresholding:} SNPs are included in the polygenic score if they are significant enough in order to reduce noice in the score. [TODO: refaire]}\label{fig:gwas2}
\end{figure}

\begin{figure}[htb]
\centerline{\includegraphics[width=0.5\textwidth]{indirect-association.png}}
\caption{Illustration of an indirect association with a phenotype due to Linkage Disequilibrium between SNPs. Source: \cite{astle2009population}.}\label{fig:gwasLD}
\end{figure}

\subsubsection{PRS for epidemiology}

[FIGURE WRAY 2014]

Polygenic analysis methods were central in demonstrating that the first phase of GWAS were underpowered, which propelled the drive for larger sample sizes that is now starting to pay off \cite[]{wray2014research}.

\subsubsection{The differing goals of association testing and risk prediction}


[COMMENCER PAR GWSignificant PUIS PLUS QUE CA (EPIDEMIOLOGY)]

[CONCLUDE WITH The differing goals of association testing and risk prediction]


\subsection{Genomic prediction}

\subsubsection{Heritability and missing heritability}

The basic components of disease risk are usually broken down into genetic susceptibility, environmental exposures and lifestyle factors. Thus, all disease incidence cannot be predicted by genetic factors only.
For a quantitative phenotype, we call heritability ($h^2$) the proportion of phenotypic variation that is attributable to genetic factors \cite[]{visscher2008heritability}.
Methods now enable the estimation of chip-heritability (also called SNP-heritability: $h^2_{SNP}$) using linear mixed models and residual maximum likelihood. For example, for a chip of 300K SNPs, it was shown that those SNPs could account for 45\% of the variance of height \cite[]{yang2010common}.
Note that the heritability of height is estimated to be around 80\% \cite[]{silventoinen2003determinants,visscher2006assumption}; the difference between these two values can be explained by the fact that 300K SNPs cannot capture the same variation in height as the 3 billion base pairs of DNA. This difference can also reflect an overestimation of heritability.

So, basically, heritability is the upper bound in terms of prediction power (when measured with $R^2$) that we can get using a model from genetic variants only.
The difference between $R^2$ and $h^2$ has been termed ``missing heritability'' \cite[]{manolio2009finding}. So, the main goal of genomic prediction and my thesis is to get best possible predictions based on genetic data in order to reduce this missing heritability.

First GWAS found only 12 associated SNPs for type 2 diabetes and only 2 for protaste cancer, explaining a small part of heritability of these diseases \cite[]{jakobsdottir2009interpretation}. Likewise, in 2008, only 40 genome-wide-significant SNPs had been identified for height, and together these explained about 5\% of heritability \cite[]{manolio2009finding}. In 2014, the number of associated SNPs had increased to around 700, explaining 20\% of heritability \cite[]{wood2014defining}.
Since most of the identified associated SNPs have effect size close to the limit dictated by the power of the studies, a likely explanation, at least in part, is that there are many common polymorphisms with effects that were too small to pass the stringent significance thresholds \cite[]{wray2008prediction}.
Therefore, as results from multiple GWAS are combined, a larger fraction of the genetic variance is likely to be explained and accurate prediction of genetic risk to disease will become possible even though the risks conveyed by individual variants are small \cite[]{wray2008prediction}.
%Therefore, sample size is not large enough to detect and estimate SNP effects precisely enough, but we can hope that with increasing sample sizes, more variants will be detected and more heritability could be explained by genetic variants.


\subsubsection{Methods for genomic prediction}


[TODO: REFAIRE]

On the contrary, the second approach does not use univariate summary statistics but instead train a multivariate model on all the SNPs and covariates \textit{at once}, optimally accounting for correlation between predictors \cite[]{Abraham2012}. The currently available models are very fast sparse linear and logistic regressions. These models include lasso and elastic-net regularizations, which reduce the number of  predictors (SNPs) included in the predictive models \cite[]{Friedman2010,Tibshirani1996,Zou2005}. Package bigstatsr provides a fast implementation of these models by using efficient rules to discard most of the predictors \cite[]{Tibshirani2012}. The implementation of these algorithms is based on modified versions of functions available in the R package biglasso \cite[]{Zeng2017}. These modifications allow to include covariates in the models, to use these algorithms on the special type of FBM called ``FBM.code256'' used in bigsnpr and to remove the need of choosing the regularization parameter.

\subsubsection{Objective and main difficulties of the thesis}


\newpage

\bibliographystyle{natbib}
\bibliography{refs}

\end{document}



\chapter[R packages for analyzing genome-wide data]{Efficient analysis of large-scale genome-wide data with two R packages: bigstatsr and bigsnpr}

%% LyX 1.3 created this file.  For more info, see http://www.lyx.org/.
%% Do not edit unless you really know what you are doing.
\documentclass[english, 12pt]{article}
\usepackage{times}
%\usepackage{algorithm2e}
\usepackage{url}
\usepackage{bbm}
\usepackage[T1]{fontenc}
\usepackage[latin1]{inputenc}
\usepackage{geometry}
\geometry{verbose,letterpaper,tmargin=2.5cm,bmargin=2.5cm,lmargin=2.5cm,rmargin=2.5cm}
\usepackage{rotating}
\usepackage{graphicx}
\usepackage{amsmath, amsthm, amssymb}
\usepackage{setspace}
\usepackage{lineno}
\usepackage{hyperref}
\usepackage{bbm}

%\usepackage{xcolor,framed}
%\colorlet{shadecolor}{blue!10}
%\begin{shaded}blabla\end{shaded}

\usepackage{pdfpages}

%\usepackage{xr}
%\externaldocument{PRS-supp}

%\linenumbers
%\doublespacing
%\usepackage[authoryear]{natbib}
\usepackage{natbib} \bibpunct{(}{)}{;}{author-year}{}{,}

%Pour les rajouts
\usepackage{xcolor}

\usepackage{dsfont}
\usepackage[warn]{textcomp}
\usepackage{adjustbox}
\usepackage{multirow}
\usepackage{graphicx}
\graphicspath{{figures/}}
\DeclareMathOperator*{\argmin}{\arg\!\min}

\let\tabbeg\tabular
\let\tabend\endtabular
\renewenvironment{tabular}{\begin{adjustbox}{max width=0.75\textwidth}\tabbeg}{\tabend\end{adjustbox}}

\makeatletter

%%%%%%%%%%%%%%%%%%%%%%%%%%%%%% LyX specific LaTeX commands.
%% Bold symbol macro for standard LaTeX users
%\newcommand{\boldsymbol}[1]{\mbox{\boldmath $#1$}}

%% Because html converters don't know tabularnewline
\providecommand{\tabularnewline}{\\}
\definecolor{clumping}{HTML}{38761D}
\definecolor{thresholding}{HTML}{1515FF}
%<span style="color:#38761D">Clumping</span> + <span style="color:#1515FF">Thresholding</span>

\usepackage{babel}
\makeatother

\begin{document}

\section{Summary of the article}

\subsection{Introduction}

GWAS data has rapidly grown due to the reduction in geneotyping costs and the imputation of many non-genotyped SNPs. Thus in 2007, there were datasets with 2000 cases and 3000 controls, genotyped over 300K. Now, we have datasets of 500K individuals, genotyped over 800K SNPs, and imputed over 90M SNPs.
Genotype data are the first data of the omics family to have grown that large. To analyze these datasets, software have been consistently procuded or updated in order to keep up with growing sizes. I think this is one rare field we some people have made their career over developing software that are now central to the community.
[FOIREUX?]
An obvious example in genetics is PLINK, a command line software whose first version has been cited aroung 17K times since 2007 and second version have already been cited more than 1500 times since 2015. This software is useful for file conversions as well as many types of analyses and is used in plant, animal and human genetics alike. 

\subsection{Methods}

\subsection{Results}

\subsection{Discussion}

\section{Article}

The following article is published in \textit{Bioinformatics}	\footnote{\url{https://doi.org/10.1093/bioinformatics/bty185}}. Supplementary data are available at \textit{Bioinformatics} online.

\includepdf[pages=-]{paper1.pdf}

\end{document}



\chapter[Efficient penalized regression for PRS]{Efficient Implementation of Penalized Regression for Genetic Risk Prediction}

\section{Summary of the article}

\subsection{Introduction}

``Clumping+Thresholding'' (C+T) is the most common method to derive Polygenic Risk Scores (PRS). C+T uses only GWAS summary statistics with a (small) individual-level data reference panel to account for linkage disequilibrium (LD). 
However, previous work showed that jointly estimating SNP effects has the potential to substantially improve predictive performance of PRS as compared to C+T \cite[]{abraham2013performance}.
Moreover, now that large individual-level datasets such as the UK Biobank are available, it would be a waste of information to not use them to their full potential \cite[]{bycroft2017genome}.
Indeed, in order for PRS to be useful in clinical settings, it should be as predictive as possible.

\subsection{Methods}

We included some efficient implementation of penalized (linear and logistic) regressions in R package bigstatsr. 
This implementation is not specific to genotype data at all, but this paper focuses on its application to predicting disease status based on large genotype data.
We recall that bigstatsr uses some matrix format stored on disk instead of memory, so that functions of this package can be very memory efficient.
To make the algorithm very efficient, we based our implementation on existing implementations that use mathematical rules to quickly discard many variables as they will not enter the final model \cite[]{tibshirani2012strong}.
These rules can be used when fitting penalized regression with either lasso or elastic net regularizations.
To facilitate the choice of the two hyper-parameters of the elastic net regularization, we develop a procedure that we call Cross-Model Selection and Averaging (CMSA).
CMSA is somehow similar to cross-validation but allows to derive an early stopping criterion that further increases the efficiency of the implementation.

We compare the penalized regressions with C+T and another method based on decision trees. We use extensive simulations to compare methods for different disease architectures, sample sizes and number of variables. We also compare methods in models with non-additive effects and show how to extend penalized regression to account for recessive and dominant effects on top of additive effects. Finally, we compare performance of methods using the UK Biobank, training models on 350K individuals and using 656K genotyped SNPs.

\subsection{Results}

We show that penalized regressions can provide large improvements in predictive performance as compared to C+T. When SNP effect sizes are small and sample size is small compared to the number of SNPs, PLR performs worse than C+T, but all methods provide poor predictive performance (AUC lower than 0.6) in this context.
In contrast, when sample size is large enough, when there are some moderately large effects, or when there are some correlation between causal variants, using PLR substantially improves predictive performance as compared to C+T.
By using some feature engineering technique, we are able to capture not only additive effects, but also recessive and dominant effects in penalized regressions.
Finally, we show that our implementation of penalized regressions is scalable to datasets such as the UK Biobank, including hundreds of thousands of both observations and variables.

\subsection{Discussion}

In this paper, we demonstrate the feasibility and relevance of using penalized regressions for PRS computation when large individual-level datasets are available. Indeed, first, we show that the larger is the data, the larger is the gain in predictive performance of PLR over C+T. Second, we show that our implementation of PLR is scalable to very large datasets such as the UK Biobank.
We discuss what makes our implementation scalable to very large datasets by explaining the algorithm and its memory requirements.
Computation time is a function of the sample size and the number of variables with a predictive effect.


\section{Article 2 and supplementary materials}

The following article is published in \textit{Genetics}	\footnote{\url{https://doi.org/10.1534/genetics.119.302019}}.

\includepdf{paper2.pdf}
\includepdf{paper2-supp.pdf}



\chapter[Making the most of Clumping and Thresholding]{Making the most of Clumping and Thresholding for polygenic scores}

\section{Summary of the article}

\subsection{Introduction}

Most of the time, only summary statistics for a GWAS dataset are available, i.e.\ the estimated effect sizes and p-values for each variant of the dataset. Because of the availability of such data en masse, specific methods using those summary data have been developed for a wide range of applications \cite[]{pasaniuc2014fast,vilhjalmsson2015modeling,bulik2015ld,pasaniuc2017dissecting,speed2018sumher}. Moreover, methods using summary statistics data are usually fast and easy to use, making them even more appealing to researchers.
One of these summary statistics based methods applicable for polygenic prediction is Clumping and Thresholding (C+T).
When on limited sample size of individual-level data are available (as opposed to summary statistics), C+T provides a competitive method for deriving predictive polygenic risk scores \cite[]{prive2019efficient}.

C+T is the simplest and most widely-used method for constructing PRS based on summary statistics and has been used for many years now. The idea behind C+T is simple because it directly uses weights learned from GWAS; it further removes SNPs as one does when reporting hits from GWAS, i.e.\ only SNPs that pass the genome-wide threshold (p-value thresholding) and that are independent association findings (clumping) are reported.
In GWAS, it is commonly accepted to use a p-value threshold of $5 \times 10^{-8}$ when reporting significant findings, yet for prediction purposes, including less significant SNPs can substantially improve predictive performance \cite[]{purcell2009common}.

Therefore, when using C+T, one has to choose a p-value threshold that balances between removing informative variants when using a stringent p-value threshold and adding too much noise in the score by including too many variants with no effect. The clumping step aims
at removing redundancy in included effects that is simply due to linkage disequilibrium (LD) between variants. Yet, clumping may as well remove independently predictive variants in nearby regions; to balance this, C+T uses as hyper-parameter a threshold on correlation between variants included. 
Thus, C+T users must choose hyper-parameters of C+T well if they want to maximize predictive performance of the polygenic score derived.
Usually, users use some default value for these parameters, expect for the p-value threshold, for which they look at different values and choose the one maximizing predictive ability in some training set.

\subsection{Methods}

We implement an efficient way to compute many C+T scores corresponding to many different sets of hyper-parameters for C+T. This is now part of R package bigsnpr \cite[]{prive2018efficient}. 
The 4 parameters we vary are the correlation threshold of clumping, the window size for looking at correlation, the p-value threshold and the imputation accuracy threshold when using imputed variants.
In total, we investigate 5600 different sets of hyper-parameters for C+T.

We also derive a C+T score for each chromosome separately, resulting in 123,200 different scores.
We propose to use stacking, i.e.\ we fit a penalized regression of these scores and learn an optimal linear combination of those scores instead of only choosing the best one \cite[]{breiman1996stacked}.
We hypothesize that Stacked Clumping and Thresholding (SCT) has the potential to make C+T more flexible and to increase its predictive performance.
Moreover, SCT results in a linear model from which we can derive an unique vector of coefficients to be used for testing in unseen individuals.

\subsection{Results}

We test 6 different simulation scenarios using the UK Biobank dataset. We also derive PRS for 8 common diseases using external summary statistics from published GWAS and dividing the UK Biobank data into training and test sets.
Investigating more hyper-parameters for C+T (we call this maxCT) instead of using standard values for these hyper-parameters (we call this stdCT) consistently improves predictive performance in simulations and real data applications.
This makes C+T competitive to state-of-the-art methods like lassosum \cite[]{mak2017polygenic}.
Moreover, SCT often provides substantial predictive performance improvement over maxCT by using different weights from those reported from the GWAS.

\subsection{Discussion}

We provide an efficient way to compute C+T scores for many different hyper-parameters values in R package bigsnpr. We show that fine-tuning hyper-parameters of C+T improves its predictive performance as compared to using some default values for clumping. Investigating 8 different disease phenotypes, we show that the optimal C+T hyper-parameters for those traits are very different, probably because these diseases have different architectures.

Instead of choosing one set of hyper-parameters that maximizes predictive performance in a training set, we propose instead to learn a combination of many C+T scores, corresponding to different sets of hyper-parameters.
This extension of C+T that we call SCT (Stacked C+T) makes C+T more flexible.
Moreover, we implement the possibility for an user of SCT to define their own groups of variants. This opens many possibilities for SCT. For example, we could derive and stack C+T scores for two related but different GWAS summary statistics, we could use external information such as functional annotations, or we could learn to differentiate between two genetically different phenotypes with similar symptoms such as type 1 and type 2 diabetes.


\section{Article and supplementary materials}

The following article is available as a preprint in \textit{bioRxiv}	\footnote{\url{https://doi.org/}}.



\chapter{Conclusion and Discussion}

%% LyX 1.3 created this file.  For more info, see http://www.lyx.org/.
%% Do not edit unless you really know what you are doing.
\documentclass[english,12pt,a4paper,twoside]{article}
\usepackage{times}
%\usepackage{algorithm2e}
\usepackage{url}
\usepackage{bbm}
\usepackage[T1]{fontenc}
\usepackage[latin1]{inputenc}
\usepackage{geometry}
%\geometry{verbose,letterpaper,tmargin=2.5cm,bmargin=3cm,lmargin=3cm,rmargin=3cm}
\usepackage{rotating}
\usepackage{graphicx}
\usepackage{amsmath, amsthm, amssymb}
\usepackage{setspace}
\usepackage{lineno}
\usepackage{hyperref}
\usepackage{bbm}

%\usepackage{xcolor,framed}
%\colorlet{shadecolor}{blue!10}
%\begin{shaded}blabla\end{shaded}

%\usepackage{xr}
%\externaldocument{PRS-supp}

%\linenumbers
%\doublespacing
\onehalfspacing
%\usepackage[authoryear]{natbib}
\usepackage{natbib} \bibpunct{(}{)}{;}{author-year}{}{,}


%Pour les rajouts
\usepackage{xcolor}

\usepackage{dsfont}
\usepackage[warn]{textcomp}
\usepackage{adjustbox}
\usepackage{multirow}
\usepackage{graphicx}
\graphicspath{{figures/}}
\DeclareMathOperator*{\argmin}{\arg\!\min}

\let\tabbeg\tabular
\let\tabend\endtabular
\renewenvironment{tabular}{\begin{adjustbox}{max width=0.75\textwidth}\tabbeg}{\tabend\end{adjustbox}}

\makeatletter

%%%%%%%%%%%%%%%%%%%%%%%%%%%%%% LyX specific LaTeX commands.
%% Bold symbol macro for standard LaTeX users
%\newcommand{\boldsymbol}[1]{\mbox{\boldmath $#1$}}

%% Because html converters don't know tabularnewline
\providecommand{\tabularnewline}{\\}
\definecolor{clumping}{HTML}{38761D}
\definecolor{thresholding}{HTML}{1515FF}
%<span style="color:#38761D">Clumping</span> + <span style="color:#1515FF">Thresholding</span>

\usepackage{babel}
\makeatother


\begin{document}

\section{Conclusion and Discussion}

\subsection{Summary of my work}

\subsection{Problem of generalization}

Polygenic Risk Scores (PRS) might become a central part in precision medicine. For now, predictive performance for most complex diseases are not good enough to be used in clinical settings. 
A major concern with PRS at the moment is their problem of generalization / transferability in different populations. 
Indeed, most GWAS have included European people only (Figure \ref{fig:GWAS-ancestry}). In 2009, 96\% of individuals included in GWAS datasets were of European ancestry \cite[]{need2009next}. In 2016, still more than 80\% of those individuals were of European descent, with an increase of the inclusion of non-European participants, mostly constituted of Asian people \cite[]{popejoy2016genomics}. People from Hispanic or African ancestry are still poorly represented \cite[]{martin2019clinical}.
This poor heterogeneity in inclusion can be explained by the fact that the more diverse are the population in the data we analyze, the more possible confounders there are to account for in order to avoid spurious results. 

\begin{figure}[htpb]
\centerline{\includegraphics[width=0.6\textwidth]{GWAS-ancestry.jpg}}
\caption{Proportion of GWAS participants by ancestry. Most GWAS include mainly European people, some now include Asian people, but other ethnicities are still poorly represented. Source: \cite{popejoy2016genomics}.}
\label{fig:GWAS-ancestry}
\end{figure}

This lack of heterogeneity in inclusion of diverse populations has several problems. First,
there are some SNP ascertainment bias because SNPs that are more common are more likely to be discovered in GWAS so that discoveries in European tend to have larger frequencies than in other populations, due to the winner's curse. If effects have a frequency that is different between populations, using these effects naturally introduces some shift in PRS distributions for different populations.
Second, rare variants are missed in GWAS if they are specific to some population that is not included in the association study \cite[]{martin2019clinical}. Thus, this limit the predictive ability of PRS in different populations to the one(s) included in the GWAS.
Third, it is accepted that genotyped SNPs, or even imputed SNPs, that are discovered in GWAS may not be the true functional SNPs (fSNPs) having an effect on disease susceptibility. Instead, GWAS are assumed to discover SNPs that tag fSNPs (tagSNPs), i.e.\ are correlated with fSNPs. Yet, LD can be different between populations so that a tagSNP can have a different correlation with the corresponding fSNP, or there can be different fSNPs for different populations. Thus, effects of these SNPs can be different and are often diluted toward zero for populations not included in the GWAS \cite[]{carlson2013generalization}. 
In conclusion, for many reasons, magnitude and frequency of effects can vary considerably between populations, and these differences are larger when populations are more genetically distant such as African population with either European or Asian populations.
These differences in prediction between populations are two-fold (Figures \ref{fig:dist-shift} and \ref{fig:pop-pred}): distributions of PRS are shifted and prediction within each distribution is also reduced \cite[]{vilhjalmsson2015modeling,martin2019clinical}.

\begin{figure}[htpb]
\centerline{\includegraphics[width=0.6\textwidth]{pred-pops.jpg}}
\caption{Distributions of Polygenic Risk Scores (PRS) for many populations and phenotypes (T2D: Type 2 Diabetes). Source: \cite{martin2017human}.}
\label{fig:dist-shift}
\end{figure}

\begin{figure}[htpb]
\centerline{\includegraphics[width=0.7\textwidth]{pop-pred-reduced.png}}
\caption{Prediction accuracy relative to European-ancestry individuals across 17 quantitative traits and 5 continental populations in the UK Biobank data \cite[]{bycroft2017genome}. Source: \cite{martin2019clinical}.}
\label{fig:pop-pred}
\end{figure}


Several solutions have been proposed to partially correct for the differences of prediction between populations. First, \cite{martin2017human} proposed to mean-center PRS for each population, yet this would require an accurate way to assess ancestry and would not work for admixed people, e.g.\ one person with an African father and an European mother \cite[]{reisberg2017comparing}.
Second, it has been suggested to include more diverse population in GWAS \cite[]{pulit2010multiethnic}. Indeed, new associations can be found if the frequency is higher in an under-represented population. 
It would be also possible to fine-map fSNPs in common for multiple populations so that their effects generalize better to any population, irrespective of LD \cite[]{carlson2013generalization,wojcik2018page}.
Finally, statistical methods are starting to emerge in order to use large European GWAS in conjunction with smaller data from another population in order to leverage both the discoveries from the large dataset and the specificities of the smaller dataset \cite[]{marquez2017multiethnic,coram2017leveraging}.


\subsection{Looking for missing heritability in rare variants}

Missing heritability, i.e.\ the gap between heritability estimations from current GWAS studies and from family studies, could reside in rare variants.
Indeed, for height and colorectal cancer, it has been shown that estimations of heritability from GWAS data could recover almost all heritability when a large proportion of low-frequency variants was present in the data \cite[]{yang2015genetic,huyghe2019discovery,wainschtein2019recovery}.
However, actual findings of significantly associated variants of low-frequency are scarce. 
For example, a GWAS of height including more than 700K individuals found 83 associated variants with allele frequencies between 0.1\% and 4.8\%, with effects up to 2 cm per allele \cite[]{marouli2017rare}. Yet, these 83 variants together accounts for only 1.7\% of the total heritability of height. 
In other large studies, one for coronary artery disease and one for type 2 diabetes, there was little evidence of low-frequency variants with large effects \cite[]{nikpay2015comprehensive,fuchsberger2016genetic}.

Associations of rare variants with traits are difficult to find for two reasons. 
First, it is very difficult to impute low-frequency variants with a good quality if using for example a small reference panel such as the 1000 genomes \cite[]{nikpay2015comprehensive}. There is now a reference panel of 32,000 individuals that is used to accurately impute variants with allele frequencies as low as 0.1\% \cite[]{mccarthy2016reference}. 
This large reference panel is European specific, which means that imputing data from other ancestries is more difficult. This is a problem because, one way to discover and accurately estimate the effect of a rare variant is to find it in a population in which its allele frequency is larger. 
Indeed, the power of association studies is dependent on the variance explained by a locus; for example, for a disease that affects 1\% of the population, we have the same power to detect a risk locus of 50\% frequency and odds ratio of 1.1 as we do for a risk locus of 0.1\% frequency and odds ratio of 2.9 \cite{wray2018common}.

The second reason is that sequencing technologies are more expensive than genotypying and imputation. Currently, studies have mostly focused on whole exome sequencing (WES) because it is cheaper than whole genome sequencing (WGS). The exome is also where the effect sizes of variants are expected to be larger and where discoveries are likely to be more immediately actionable \cite[]{zuk2014searching}. Yet, sample sizes of sequencing studies remain small and special considerations and challenges arise when testing rare frequency variants from these studies \cite[]{auer2015rare}.
Thus, sample size is the limiting factor in variant discovery, not genotyping technology \cite{wray2018common}. 
It is probably the limiting factor in prediction too.

\subsection{Looking for missing heritability in non-additive effects}

Knowledge about biological pathways and gene networks implies that epistasis (gene interactions) might be important to consider \cite{hill2008data}. 
More than explaining missing heritability, genetic interactions could also create phantom heritability, i.e.\ could make current estimation of heritability upward biased \cite[]{zuk2012mystery}.
There have been some findings of interaction between loci, but mainly for autoimmune diseases where there are strong effects around genes of chromosome 6 \cite[]{lenz2015widespread,goudey2017interactions}.
Yet, these interaction effects explain little to phenotypic variance as compared to additive effects \cite[]{lenz2015widespread}.
In general, data and theory point to mainly additive genetic variance \cite{hill2008data}.

Moreover, interactions are challenging to find for two reasons, and dedicated methods to epistasis detection have been implemented \cite[]{niel2015survey}. First, it is analytically impractical to search for such interaction effects because it would require testing more than 100 billion pairs of variants, even for a small genotyping array. Second, because of this huge number of tests, correction for multiple testing allows the detection of highly significant interactions only.

Finally, even if we find such interaction effects, they are unlikely to dramatically improve risk prediction for complex diseases, but could still provide insights into their ethiology \cite[]{aschard2012inclusion}. 
Moreover, due to differences in effect sizes and LD between populations, epistatic effects are even more unlikely than additive effects to replicate to different populations  \cite{hill2008data}.

\subsection{Integration of multiple data sources}

[TODO: FINISH/MODIFY]

There are many genetic data out there. Some large individual-level data such as the UK biobank are available \cite[]{bycroft2017genome}. When GWAS data is not publicly available, summary statistics are often publicly shared instead. 
Usually, predictive models are based on either individual-level data (e.g.\ penalized regression) or summary statistics (e.g.\ C+T).
Building models that combine both individual-level data and summary statistics, possibly including different populations, is necessary to increase predictive power.
We started to do this by implementing the SCT method in our latest paper, where we combine several summary statistics based predictors using large individual-level data.
Alike with the adaptive lasso \cite[]{zou2006adaptive}, one could also think of penalizing SNPs differently in individual-level data methods, applying a penalization factor to each SNP based on their significance in external summary statistics.

Human diseases are inherently complex and governed by the complicated interplay of several underlying factors \cite[]{dey2013integration}.
For a trait or a disease, prediction based on genetic data only is ultimately capped by heritability.
Therefore, prediction must integrate other types of data if we want to predict beyond the limit of heritability (Figure \ref{fig:data-layers}).
Yet, integrating variables with different formats, types, structure, dimensionality and missing values is a challenging problem \cite[]{dey2013integration}.
One could integration genetic data with clinical data. For example, \cite{inouye2018genomic} designed a polygenic risk score (PRS) with higher discrimitative ability for coronary artery disease than any of 6 conventional risk factors (smoking, diabetes, hypertension, body mass index, high cholesterol and family history).
Using this PRS with all 6 conventional risk factors increases discriminative ability as compared to using the PRS only or the 6 factors only.
Electronic health records (EHR) make possible to integrate large biobank datasets with large clinical, environmental and phenotypic information \cite[]{roden2016integrating}.
%Three different approaches for data integration are presented in figure \ref{fig:int-data} \cite[]{dey2013integration,zitnik2019machine}.

\begin{figure}[htpb]
\centerline{\includegraphics[width=0.7\textwidth]{data-layers}}
\caption{Geographic information system of a human being. The different layers of data available for an individual. Source: \cite{topol2014individualized}.}
\label{fig:data-layers}
\end{figure}


\subsection{Future work}

I will probably continue to work in the field of Predictive Human Genetics. I am currently visiting the National Center for Register-based Research (NCRR) in Aarhus, Denmark. Researchers there are mostly epidemiologists using some national registers where they have information on all Danes over decades. Most of their work is funded to look at psychiatric disorders and they are now interested in how genetics influence psychiatric conditions. It is a good opportunity to work on a large national biobank dataset with Bjarni Vilhj\'almsson.


\newpage

\bibliographystyle{natbib}
\bibliography{refs}

\end{document}



\cleardoublepage
\phantomsection
\addcontentsline{toc}{chapter}{Bibliography}
\bibliographystyle{natbib}
\bibliography{refs}


\appendix

\chapter{Code optimization based on linear algebra}

%% LyX 1.3 created this file.  For more info, see http://www.lyx.org/.
%% Do not edit unless you really know what you are doing.
\documentclass[english, 12pt]{article}
\usepackage{times}
%\usepackage{algorithm2e}
\usepackage{url}
\usepackage{bbm}
\usepackage[T1]{fontenc}
\usepackage[latin1]{inputenc}
\usepackage{geometry}
\geometry{verbose,letterpaper,tmargin=2.5cm,bmargin=2.5cm,lmargin=2.5cm,rmargin=2.5cm}
\usepackage{rotating}
\usepackage{color}
\usepackage{graphicx}
\usepackage{amsmath, amsthm, amssymb}
\usepackage{setspace}
\usepackage{lineno}
\usepackage{hyperref}
\usepackage{bbm}

%\usepackage{xcolor,framed}
%\colorlet{shadecolor}{blue!10}
%\begin{shaded}blabla\end{shaded}

\linenumbers
\doublespacing
%\usepackage[authoryear]{natbib}
\usepackage{natbib} \bibpunct{(}{)}{;}{author-year}{}{,}

%Pour les rajouts
\usepackage{color}
\definecolor{trustcolor}{rgb}{0,0,1}

\usepackage{dsfont}
\usepackage[warn]{textcomp}
\usepackage{adjustbox}
\usepackage{multirow}
\usepackage{graphicx}
\graphicspath{{figures/}}
\DeclareMathOperator*{\argmin}{\arg\!\min}

\let\tabbeg\tabular
\let\tabend\endtabular
\renewenvironment{tabular}{\begin{adjustbox}{max width=\textwidth}\tabbeg}{\tabend\end{adjustbox}}
\usepackage{numprint}
\npthousandsep{,}

\makeatletter

%%%%%%%%%%%%%%%%%%%%%%%%%%%%%% LyX specific LaTeX commands.
%% Bold symbol macro for standard LaTeX users
%\newcommand{\boldsymbol}[1]{\mbox{\boldmath $#1$}}

%% Because html converters don't know tabularnewline
\providecommand{\tabularnewline}{\\}

\usepackage{babel}
\makeatother


\begin{document}

\section{Appendix}

\subsection{Lightning fast multiple association testing}.

In this section, I describe how to quickly test many variables for an association with a continuous outcome of interest. For example, let us make a Genome-Wide Association Study (GWAS) of height, i.e. we want to determine which genome variants are associated with height.

The model we want to test is $$y = \beta s + X \gamma + \epsilon~,$$ where $s$ is one variant (we want to do this for each variant, separately), $X$ are some covariates to adjust for some possible confounding factors (a matrix of $N$ samples over $K$ columns, including a column of $1$s to account for an intercept in the model). 
We are only interested in estimating $\hat{\beta}$ and computing a p-value corresponding to the significance of the alternative hypothesis that $\beta \neq 0$.

\cite{sikorska2013gwas} show that we can rewrite this problem as $$y^* = \beta s^* + \epsilon~,$$ where $y^* = y - X (X^T X)^{-1} X^T y$ and $s^* = s - X (X^T X)^{-1} X^T s$. Thus, this becomes a simple linear problem which is easy and fast to solve. We have
\begin{align*}
\hat{\beta} &= \dfrac{s^{*T} y^*}{s^{*T} s^*} \\
\widehat{\text{var}}(\hat{\beta}) &= \dfrac{(y^* - \hat{\beta} s^*)^T (y^* - \hat{\beta} s^*)}{(N - K - 1) ~ s^{*T} s^*} \\
\frac{\hat{\beta}}{\sqrt{\widehat{\text{var}}(\hat{\beta})}} &\sim T(N - K - 1)
\end{align*}

We go further by computing the singular value decomposition $X = U \Delta V^T$ ($N \times K$ matrix). As $N \gg K$, we have $U^T U = I_K$, $V^T V  = I_K$ and $V V^T = I_K$. Thus $X (X^T X)^{-1} X^T = U \Delta V^T (V \Delta U^T U \Delta V^T)^{-1} V \Delta U^T = U \Delta V^T (V \Delta^2 V^T)^{-1} V \Delta U^T = U \Delta V^T (V \Delta^{-2} V^T) V \Delta U^T = U U^T$.
We can simplify $s^{*T} y^* = (s - U U^T s)^T y^* = s^T y^* - s^T \underbrace{U U^T y^*}_0 = s^T y^*$, 
$s^{*T} s^* = (s - U U^T s)^T (s - U U^T s) = s^T s - 2 s^T U U^T s + s^T U U^T U U^T s = s^T s - s^T U U^T s = s^T s - z^T z$, where $z = U^T s$, 
and $(y^* - \hat{\beta} s^*)^T (y^* - \hat{\beta} s^*) = y^{*T} y^* - 2 \hat{\beta} s^{*T} y^* + \hat{\beta}^2 s^{*T} s^* = y^{*T} y^* - 2 \hat{\beta} s^{*T} y^* + \hat{\beta} s^{*T} y^* = y^{*T} y^* - \hat{\beta} s^{T} y^*$.
So, we only need to compute
\begin{align*}
z &= U^T s \\
\hat{\beta}_{\text{num}} &= s^{T} y^* \\
\hat{\beta}_{\text{deno}} &= s^T s - z^T z \\
\hat{\beta} &= \hat{\beta}_{\text{num}} / \hat{\beta}_{\text{deno}} \\
\widehat{\text{var}}(\hat{\beta}) &= \dfrac{y^{*T} y^* - \hat{\beta} ~ \hat{\beta}_{\text{num}}}{(N - K - 1) ~ \hat{\beta}_{\text{deno}}}~.
\end{align*}
Since $U$ and $y^*$ are computed only once for all variants, you can apply those formulas to compute these statistics for \numprint{1000000} variants and $N$=\numprint{500000} samples and $K$=$11$ covariates in one hour only \cite[]{prive2018efficient}. This is implemented in function big\_univLinReg() of package bigstatsr.


%%%%%%%%%%%%%%%%%%%%%%%%%%%%%%%%%%%%%%%%%%%%%%%%%%%%%%%%%%%%%%%%%%%%%%%%%%%%%%%%



\newpage

\bibliographystyle{natbib}
\bibliography{refs}

\end{document}


\afterpage{\blankpage}


\end{document}
